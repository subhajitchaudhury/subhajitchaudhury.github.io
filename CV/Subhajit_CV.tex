\documentclass[margin,line]{res}
\usepackage[utf8]{inputenc}
\usepackage[english]{babel}
\usepackage{hyperref}
\usepackage{etoolbox,refcount}
\usepackage{multicol}
\hypersetup{
	colorlinks=true,
	linkcolor=blue,
	filecolor=magenta,      
	urlcolor=blue,
}

\oddsidemargin -.5in
\evensidemargin -.5in
\textwidth=6.0in
\itemsep=0in
\parsep=0in
% if using pdflatex:
%\setlength{\pdfpagewidth}{\paperwidth}
%\setlength{\pdfpageheight}{\paperheight} 

\newenvironment{list1}{
  \begin{list}{\ding{113}}{%
      \setlength{\itemsep}{0in}
      \setlength{\parsep}{0in} \setlength{\parskip}{0in}
      \setlength{\topsep}{0in} \setlength{\partopsep}{0in} 
      \setlength{\leftmargin}{0.17in}}}{\end{list}}
\newenvironment{list2}{
  \begin{list}{$\bullet$}{%
      \setlength{\itemsep}{0in}
      \setlength{\parsep}{0in} \setlength{\parskip}{0in}
      \setlength{\topsep}{0in} \setlength{\partopsep}{0in} 
      \setlength{\leftmargin}{0.2in}}}{\end{list}}


\begin{document}

%\name{\LARGE{Subhajit Chaudhury} \hspace*{2.65in} \LARGE{Curriculum vitae}}
\name{\LARGE{Subhajit Chaudhury}}
\begin{resume}
\section{}
\vspace{.05in}

\section{\sc Contact\\Information}
Corporate L'Espoir Bldg B-104,  \hfill \textbf{Phone} : +81 7021837790 \\
1 Chome 51-1 Suenaga, Kawasaki, Japan   \hfill \textbf{Email} : \href{mailto:subhajitiitb14@gmail.com}{subhajitiitb14@gmail.com} \\
\textbf{Website}: \url{https://subhajitchaudhury.github.io/}\\
%\textbf{Github}: \url{https://github.com/subhajitchaudhury}

%{\it Company Email:}   s-chaudhury@ap.jp.nec.com   \\
%{\it Personal Email:}   subhajitiitb14@gmail.com   \\
%{\it Mobile Phone:}   +81 (070) 2183 7790 \\
%{\it Website:}  \url{https://sites.google.com/site/subhaweb1411/}
\vspace{-.05in}
%\section{\sc Research Interests}
%My research interests are computer vision, machine learning and computer graphics. Recently I am working on deep learning which %is a very exciting and popular field of research. 
%\section{\sc Education}
%{\bf {Indian Institute of Technology, Bombay India}\\
%%{\em Department of Statistics} 
%\vspace*{-.1in}
%\begin{list1}
%\item[] Master of Technology in Electrical Engineering, Graduated: June 2014)
%\begin{list2}
%\vspace*{.05in}
%\item Dissertation Topic:  ``Interaction with virtual environment using kinect
%and haptic devices'' 
%%\item Dissertation Topic:  ``Hierarchical Models for Multiple Ratings
%%  in Performance-Based\\ \hspace*{1.23in} Student Assessments.'' 
%\item Advisor:  Prof. Subhasis Chaudhuri
%\end{list2}
%\vspace*{.05in}
%\item[] B.E., Electrical Engineering,  May 2012
%\end{list1}
%
%{\bf Duke University}, Durham, North Carolina USA\\
%%{\em Department of Mathematics and Statistics} 
%\vspace*{-.1in}
%\begin{list1}
%\item[] M.S., Botany (Ecology),  May, 1998
%\end{list1}
%
%{\bf Carleton College}, Northfield, Minnesota USA\\
%%{\em Department of Mathematics and Statistics} 
%\vspace*{-.1in}
%\begin{list1}
%\item[] B.A., Biology,  May, 1993
%\end{list1}
%\vspace*{-0.2in}
\section{\sc Research\\Interest}
I am working as a researcher at IBM Research-Tokyo and pursuing my Ph.D. at the University of Tokyo. My research interests lie in computer vision and reinforcement learning. At IBM, I am working on vision based imitation learning in artificial agents. In my Ph.D. work, I am focusing on exploring interpretability in neural network models for computer vision applications.

%\vspace*{-0.05in}
\section{\sc Education}

\textbf{The University of Tokyo}, Japan \hfill April 2018 - Present \\
Ph.D., Information Science and Technology\\
Advised by Prof. Toshihiko Yamasaki

\textbf{Indian Institute of Technology, Bombay}, India \hfill July 2012 - June 2014 \\
M.Tech, Department of Electrical Engineering\\
GPA : \textbf{9.81 out of 10}\\
Advised by Prof. Subhasis Chaudhuri %(\textbf{Rank: $\mathbf{2^{nd}}$/24})
%\\ \textit{Dissertation Topic}:  ``\textbf{Interaction with virtual environment using kinect and haptic devices}'' 
%\\
%\textit{Advisor}:  \textbf{Prof. Subhasis Chaudhuri}

\textbf{Jadavpur University}, India \hfill July 2008 - June 2012 \\ 
B.E.(Hons.) Department of Electrical Engineering\\
GPA - \textbf{8.90 out of 10} (\textbf{Rank: $\mathbf{3^{rd}}$/125}) \\
Advised by Prof. Amitava Chatterjee


\section{\sc Publications}

%Subhajit Chaudhury, Gaku Nakano, Jun Takada, Akihiko Iketani, \textbf{Spatial-temporal motion field analysis for crack detection on concrete surfaces}, WACV 2017(submitted)

%\textbf{1)} \textit{Subhajit Chaudhury}, Daiki Kimura, Tadanobu Inoue, Ryuki Tachibana, \textbf{Model-based imitation learning from state trajectories}, submitted to International Conference on Learning Representations (ICLR), 2018. 

\textbf{1)} Daiki Kimura, \textit{Subhajit Chaudhury},  Ryuki Tachibana and Sakyasingha Dasgupta, \textbf{\href{https://arxiv.org/abs/1806.01267}{Internal Model from Observations for Reward Shaping}}, International Conference of Machine Learning(ICML), Adaptive Learning Agents Workshop (ALA), 2018

\textbf{2)} Phongtharin Vinayavekhin, \textit{Subhajit Chaudhury},  Asim Munawar, Don Joven Agravante, Giovanni De Magistris, Daiki Kimura and Ryuki Tachibana, \textbf{\href{https://arxiv.org/abs/1806.08523}{Focusing on What is Relevant: Time-Series Learning and Understanding using Attention}}, International Conference on Pattern Recognition (ICPR), 2018

\textbf{3)} Tadanobu Inoue, \textit{Subhajit Chaudhury}, Giovanni De Magistris and Sakyasingha Dasgupta, \textbf{\href{https://arxiv.org/abs/1709.06762}{Transfer learning from synthetic to real images using variational auto-encoders for robotic applications}}, IEEE International Conference on Image Processing (ICIP), 2018

\textbf{4)} \textit{Subhajit Chaudhury}, Sakyasingha Dasgupta, Asim Munawar, Md. Salam Khan and Ryuki Tachibana, \textbf{\href{https://arxiv.org/abs/1707.00860}{Conditional generation of multi-modal data using constrained embedding space mapping}},	International Conference on Machine Learning (ICML), Implicit Models Workshop, 2017 

\textbf{5)} \textit{Subhajit Chaudhury}, Sakyasingha Dasgupta, Asim Munawar, Md. Salam Khan, Ryuki Tachibana, \textbf{\href{https://www.researchgate.net/publication/319256866_Text_to_Image_Generative_Model_Using_Constrained_Embedding_Space_Mapping}{Text to image generative model using constrained embedding space mapping}}, IEEE International Workshop on Machine Learning for Signal Processing (MLSP), 2017 (\textbf{Oral})

\textbf{6)} \textit{Subhajit Chaudhury}, Gaku Nakano, Jun Takada, Akihiko Iketani, \textbf{\href{http://ieeexplore.ieee.org/document/7926627/}{Spatial-temporal motion field analysis for crack detection on concrete surfaces}}, IEEE Winter Conference on Applications of Computer Vision (WACV) 2017

\textbf{7)} \textit{Subhajit Chaudhury} and Hiya Roy, \textbf{\href{http://ieeexplore.ieee.org/document/7986849/}{Can fully convolutional networks perform well for general image restoration problems?}}, Intl. Conf. on Machine Vision Applications, 2017

\textbf{8)} Vijay Daultani, \textit{Subhajit Chaudhury}, Kazuhisa Ishizaka, \textbf{\href{https://subhajitchaudhury.github.io/data/SASIMI_paper.pdf}{Convolutional Neural Network Layer Re-ordering for acceleration}}, 20th Workshop on Synthesis And System Integration of Mixed Information (SASIMI), Kyoto, Japan, 2016 

\textbf{9)} Sourav Saha, Pritha Ganguly, \textit{Subhajit Chaudhury}. \textbf{\href{https://subhajitchaudhury.github.io/data/icvgip_final_version.pdf}{Vision based human pose estimation for virtual cloth fitting}}. Proceedings of the 2014 Indian Conference on Computer Vision Graphics and Image Processing (ICVGIP)%\footnote{ICVGIP is a top conferences in the field of Computer Vision in India}

\textbf{10)} \textit{Subhajit Chaudhury}, Subhasis Chaudhuri, \textbf{\href{https://subhajitchaudhury.github.io/data/chaudhury2014.pdf}{Volume preserving haptic pottery}}, 2014 IEEE Haptics Symposium (HAPTICS), Houston, TX, 2014, pp.129-134. (\textbf{Oral})


\section{\sc Professional activities} 

\begin{list2}
	%\item \textbf{Neural attention-based crack detection} \hfill Oct 2016 - Present \\
	%Currently designing neural attention-based context-aware crack contour tracking using recurrent fully convolutional nets (RFCN) for local crack segmentation and fully differential read, write operation for shifting visual attention. \\
	
	
	\item Reviewer for International Conference on Robotics and Automation (ICRA), 2018 
	
	\item Reviewer for International Conference on Intelligent Robots and Systems (IROS), 2018 
	
	\item Reviewer for IEEE Transactions on Multimedia, 2018.
	
	%	\item 	GRE : 328/340 (Verbal: 158/170, Quantitative: 170/170) 
\end{list2}

\section{\sc Research experience} 

\textbf{IBM Research-Tokyo, Japan} \hfill April 2017- Present  \\
\textbf{Topic} : \textit{Vision based reinforcement learning}\\
\textbf{Position} : \textit{\textbf{Staff Researcher}, Cognitive Robot Innovation Laboratories}\\

    \begin{list2}
  	%\item \textbf{Neural attention-based crack detection} \hfill Oct 2016 - Present \\
  	%Currently designing neural attention-based context-aware crack contour tracking using recurrent fully convolutional nets (RFCN) for local crack segmentation and fully differential read, write operation for shifting visual attention. \\
  \item \textbf{Temporal segmentation of rally scenes in table tennis videos} : %\hfill Oct 2015 - Sept 2016\\
  Developed a deep learning based event detection system that segments rally scenes from table tennis videos captured in the wild from arbitrary viewing angles. Our methods demonstrate average F1-score performance above 90\% while also being robust to noise like camera shaking and occlusions.\\

  \item \textbf{Imitation learning from high dimensional observations} : %\hfill Oct 2015 - Sept 2016\\
  Developed an imitation learning method that learns action policies from high dimensional expert observations (like raw videos) in the absence of reward signal. Our method can imitate raw YouTube videos with similar performance to the case when true reward signal is available.\\

  	\item \textbf{Transfer learning from synthetic to real images using VAEs for robotic applications} :  %\hfill Oct 2015 - Sept 2016\\
  	Developed a method to transfer object detection learned in a simulation environment to the real world by performing a two-stage training on variational auto-encoders (VAE). The proposed method is 6 to 7 times more precisely than baseline methods and robust to lighting conditions.\\

  	\item \textbf{Conditional generation of multi-modal data using constrained embedding space mapping} : %\hfill June 2016 - Sept 2016\\
  	Developed a multi-modal generative method that maps multiple data modalities to a common latent space enabling simple cross modal inference. Proposed method can synthesize images from text and raw audio input while producing better PSNR values than baseline methods.\\
  	
  	%	\item 	GRE : 328/340 (Verbal: 158/170, Quantitative: 170/170) 
  \end{list2}
  

  
  \textbf{ NEC Central Research Labs, Japan } \hfill Oct 2014- March 2017  \\
  \textbf{Topic} : \textit{Deep learning based predictive infrastructure maintenance }\\
  \textbf{Position} : \textit{\textbf{Researcher}, Predictive Infrastructure Maintenance group}\\

  \begin{list2}
  	%\item \textbf{Neural attention-based crack detection} \hfill Oct 2016 - Present \\
	%Currently designing neural attention-based context-aware crack contour tracking using recurrent fully convolutional nets (RFCN) for local crack segmentation and fully differential read, write operation for shifting visual attention. \\

  	  \item \textbf{Spatial-temporal motion analysis for invisible crack detection} : %\hfill Apr 2015 - Sept 2016 \\
  	  Developed a crack detection algorithm that identifies internal cracks by finding discontinues in dense 2D motion fields using energy minimization on a Conditional Random Fields (CRF). Improved F1 score by 0.22 compared to state-of-the-art image based methods.\\

\pagebreak

  	\item \textbf{Deep learning for image-based crack detection} : %\hfill Oct 2015 - Sept 2016\\
  	Developed a fully convolutional network based system for pixel-level crack localization from raw images. Collaborated with Texas Department of Transportation (TxDOT) for application on real captured road videos with real-time performance (16fps for VGA images) with similar localization accuracy to state-of-the-art methods.\\

%\vspace*{0.15in}

  	\item \textbf{Accelerating convolutional neural nets by layer re-ordering} :  %\hfill June 2016 - Sept 2016\\
  	Obtained computational speed-up of 4x in activation units with 5\% overall improvement, in convolutional neural networks inference by rearranging pooling and activation layer ordering. \\

  	%	\item 	GRE : 328/340 (Verbal: 158/170, Quantitative: 170/170) 
  \end{list2}

%\vspace*{-0.15in}

\section{\sc Academic research \\ projects} 
%\vspace{-0.01in}

{\bf Indian Institute of Technology (IIT), Bombay}  \hfill July 2012- June 2014 \\
\textit{Master of Technology (M. Tech) thesis}, India \hfill Prof. Subhasis Chaudhuri\\

  \begin{list2}
  	%\item \textbf{Neural attention-based crack detection} \hfill Oct 2016 - Present \\
  	%Currently designing neural attention-based context-aware crack contour tracking using recurrent fully convolutional nets (RFCN) for local crack segmentation and fully differential read, write operation for shifting visual attention. \\

  	\item \textbf{Volume preserving haptic pottery} : %\hfill May 2013 - Dec 2013	 \\
  	Developed a realistic deformation model for interactive rendering of semi-solid clay based virtual pottery with volume preservation. Proposed model enabled real time visual feedback at 25 fps and tactile feedback at 1000 Hz which is much faster than prior works.\\
  	

  	\item \textbf{Feel Chat : 3D interactive virtual chat room with touch} : %\hfill Aug 2013 - June 2014\\
  	Developed a virtual reality chatting system using virtual reality headsets and wearable tactile suit where users can touch the surrounding virtual environment by tactile feedback.  \\
  	

  	\item \textbf{Web-cam based virtual trial room} : %\hfill Feb 2014 - June 2014\\
  	Developed a real-time virtual cloth fitting using generic web camera input by structurally aligning the input garment to the skeletal joints using OpenCV.\\
  
  \end{list2}
  
\vspace*{-0.15in}
{\bf Jadavpur University, India}  \hfill July 2008 - June 2012 \\
\textit{Undergraduate (B.E.) project} \hfill Prof. Amitava Chatterjee\\
	\vspace*{-0.05in}
\begin{list2}
	\vspace*{-0.05in}
	\item \textbf{Vision based door detection} : %\hfill Jan 2012 - May 2012 	 \\
	Developed a  door detection algorithm for mobile robot navigation by generating proposals for candidate door-like structures based on geometric features.\\
	%	\item 	GRE : 328/340 (Verbal: 158/170, Quantitative: 170/170) 
\end{list2}

%  	\vspace*{-0.15in}

%\section{\sc Patents} 
%%\vspace*{-2.5mm}
%\textbf{1)}  Subhajit Chaudhury and Gaku Nakano , \textbf{A device for automatic crack detection from 2D motion in video sequences}, PCT/JP2016/072700 (\textit{filed at NEC})
%%Subhajit Chaudhury and Gaku Nakano , \textbf{3D motion estimation device from 2D optical flow}, PCT/JP2016/001406 (\textit{filed at NEC})
%
%\textbf{2)}  Subhajit Chaudhury and Gaku Nakano , \textbf{A device for direct 3D deformation estimation from 2D optical flow}, PCT/JP2016/001406 (\textit{filed at NEC})
%
%\textbf{3)}  Subhajit Chaudhury, Vineet Gokhale, Subhasis Chaudhuri , \textbf{A virtual reality system and method for providing synchronous tactile feedback for user interaction}, Indian Patent, 503/MUM/2015 (\textit{filed at IIT Bombay})


\section{\sc Awards and\\ Achievements} 
\begin{list2}
	\item Secured All India Rank \textbf{33 out of 110,125} students in Electrical Engineering, GATE-2012.
	\item Secured All India Rank \textbf{125 out of 72,680} students in Electrical Engineering, GATE-2011.
	\item Secured rank \textbf{86}/\textbf{80,000} in West Bengal Joint Entrance Examination, 2008 for Engineering.
	%\item Ranked 152 (state-rank) in All India Engineering Entrance Examination (AIEEE), 2008
	%\item Former IEEE Students Member till December 2014.
	\item Received academic excellence award for $\mathbf{1^{st}}$ position in high school for both class 10 (ICSE-2006) and class 12 (ISC-2008) national board exam.
	\item Awarded $\mathbf{1^{st}}$ position prize for winning Don-Bosco Inter-School coding competition.
%	\item Achieved $\mathbf{2^{nd}}$ position in All-Bengal spelling competition.
%	\item 	GRE : 328/340 (Verbal: 158/170, Quantitative: 170/170) 
\end{list2}

%\section{\sc Other \\ Relevant\\ projects}
%\textbf{1) Can fully convolutional networks perform well for general image restoration problems?}\\
%Proposed a fully convolutional network model for learning direct end-to-end mapping between corrupted images and desired clean images for image denoising and restoration task. 
%
%\textbf{2) Direct reconstruction of dense 3D non-rigid deformation from 2D	correspondences}\\
%Developed dense 3D surface deformation estimation method from monocular images by minimizing local-global 3D to 2D motion re-projection functional using Euler-Lagrange minimization method.
%
%\textbf{3) Deep Hyper spectral image classification}\\
%Implemented classification of Hyperspectral Satellite Images Using Convolutional Neural Networks. 

%\vspace*{-.1in}

%\section{\sc Teaching\\ Experience} 
%%\begin{list2}
%%	\item TA for Signals and System, Prof. Animesh Kumar, IIT Bombay \hfill Spring-2013.
%%	\item TA for Digital Signal Processing , Prof. Animesh Kumar, IIT Bombay \hfill Autumn-2013
%%	\item TA for Computer Vision(EE702), Prof. Subhasis Chaudhuri, IIT Bombay. \hfill Spring 2014
%%\end{list2}
%
%\begin{list2}
%	%\item \textbf{Neural attention-based crack detection} \hfill Oct 2016 - Present \\
%	%Currently designing neural attention-based context-aware crack contour tracking using recurrent fully convolutional nets (RFCN) for local crack segmentation and fully differential read, write operation for shifting visual attention. \\
%	\item \textbf{ EE702: Computer Vision} \hfill Spring 2014 	 \\
%	\textit{Teaching Assistant} for \textit{Prof. Subhasis Chaudhuri}, IIT Bombay\\
%	
%	\item \textbf{ EE603: DSP and its application} \hfill Autumn-2013 	 \\
%	\textit{Teaching Assistant} for \textit{Prof. Animesh Kumar}, IIT Bombay\\
%
%	\item \textbf{ EE210: Signals and System} \hfill Spring-2013 	 \\
%	\textit{Teaching Assistant} for \textit{Prof. Animesh Kumar}, IIT Bombay\\
%	
%	%	\item 	GRE : 328/340 (Verbal: 158/170, Quantitative: 170/170) 
%\end{list2}
% 
% \vspace*{-0.15in}
  	
\section{\sc Relevant\\ Courses} 
\textbf{Electrical Engineering:} Wavelets, Statistic Signal Processing , Applied Linear Algebra , Digital Signal Processing, Number Theory and Cryptography, Digital Message Transmission

\textbf{Computer Science:} Computer Vision, Foundations of Machine Learning, Computer Graphics, Advanced Computer Graphics 
%\vspace*{-0.1in}
\section{\sc Computer\\ Skills} 
\begin{list2}
\item \textbf{Programming Languages} : \textit{Python, C++, C, Java}
\item \textbf{Tools} : \textit{Matlab, ROS, Gazebo, OpenCV, CUDA, OpenGL, }
\item \textbf{Machine learning Tools} : \textit{Tensorflow, Keras, Pytorch, scikit-learn}

\end{list2}

%\section{\sc Language Skills} 
%\begin{list2}
%	\item English: Native level proficiency (TOEFL score :111/120, TOEIC score : 990/990)
%	\item Japanese: Daily Conservation in Japanese (Japanese Language Proficiency Test - JLPT N4)
%\end{list2}


%\section{\sc Referees} 
%Available upon request
%\begin{list2}
%	\begin{multicols}{2}
%		\item 
%		Prof. Subhasis Chaudhuri \\ 
%		Deputy Director (AIA) \& Professor, \\
%		Department of Electrical Engineering,\\
%		IIT Bombay, India\\
%		email : sc@ee.iitb.ac.in
%
%		\item 
%		Prof. Amitava Chatterjee\\
%		Professor, \\
%		Department of Electrical Engineering  \\
%		Jadavpur University,Kolkata, India\\
%		email : achatterjee@ee.jdvu.ac.in
%				
%	\end{multicols}
%\end{list2}
\section{\sc Extra\\curricular\\activities} 
\begin{list2}
	\item Executive Council member of IIT Bombay Alumni Association in Tokyo (2015-2018)
	\item Passed Japanese Language Proficiency Test, N4 level
	\item Member of IIT Bombay swimming club (2012-2014)
	%\item Trained in classical guitar lessons and performed at many events.
	%\item Hobbies : Photography, music composition, cooking, hiking

\end{list2}
%\vspace*{-0.1in}

\end{resume}
\end{document}





